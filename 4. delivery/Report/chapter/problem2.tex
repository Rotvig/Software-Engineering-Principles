\chapter{Problem 2}
\label{chp:two}

\begin{enumerate}[label=(\alph*)]

\item  Verify the program

We can verify the code by introducing an invariant in the loop. The invariant must always hold.


\begin{lstlisting}

{
	assert m == m0;
	int i;
	for(i=0; i<k; i++) {
		m = m + x;
		invariant 0 <= i <= k  /\ m == m0 + (x * k)
	}
	assert m == m0 + (x * k);
}
\end{lstlisting}


\item Will the program always work correctly

If k has a lesser value than 0 and x is not equal to 0 then the invariant will not hold. Having a k lesser than zero would also make the last assert fail as the loop would terminate after zero iterations and then m would still be equal to initial value and therefore not m0 + (x*k) assuming x is not zero.


\item  Use the assertions to generate a unit test case for the “inner program”

When generating unit tests it is important to focus on boundary cases to ensure that it is possible to actually create and run the tests within some given time frame.

Code under test:
\begin{lstlisting}
{
int i;
for(i=0; i<k; i++) {
	m = m + x;
	}
}
\end{lstlisting}

So in the unit tests it must generally hold that when the loop is done m must equal m+x*k. This is the assertion we are making in certain scenarios. The test scenarios are the following
\begin{itemize}
	\item k as positive, zero and negative
	\item x as positive, zero and negative
	\item m as positive, zero and negative
\end{itemize}

\begin{lstlisting}
 [Fact]
 public void KIsPositive()
 {
 int k = 5, x = 5, m = 5;
 Assert.Equal(m+x*k, CodeUnderTest(k, x, m ));
 }
 
 [Fact]
 public void KIsNegative()
 {
 int k = -1, x = 5, m = 5;
 Assert.Equal(m + x * k, CodeUnderTest(k, x, m));
 }
 
 
 [Fact]
 public void KEqualsZero()
 {
 int k = 0, x = 5, m = 5;
 Assert.Equal(m + x * k, CodeUnderTest(k, x, m));
 }
 
 [Fact]
 public void XIsPositive()
 {
 int k = 5, x = 5, m = 5;
 Assert.Equal(m + x * k, CodeUnderTest(k, x, m));
 }
 
 [Fact]
 public void XIsZero()
 {
 int k = 5, x = 0, m = -5;
 Assert.Equal(m + x * k, CodeUnderTest(k, x, m));
 }
 
 [Fact]
 public void XIsNegative()
 {
 int k = 5, x = -5, m = -5;
 Assert.Equal(m + x * k, CodeUnderTest(k, x, m));
 }
 
 [Fact]
 public void MIsPositive()
 {
 int k = 5, x = 5, m = 5;
 Assert.Equal(m + x * k, CodeUnderTest(k, x, m));
 }
 
 [Fact]
 public void MIsZero()
 {
 int k = 5, x = 0, m = 0;
 Assert.Equal(m + x * k, CodeUnderTest(k, x, m));
 }
 
 [Fact]
 public void MIsNegative()
 {
 int k = 5, x = -5, m = -5;
 Assert.Equal(m + x * k, CodeUnderTest(k, x, m));
 }
 
 int CodeUnderTest(int k, int x, int m)
 {
 int i;
 for (i = 0; i < k; i++)
 {
 m = m + x;
 }
 
 return m;
 }
\end{lstlisting}

When we run the test we get the following
\myFigure{unittests}{asdf}{asdf}{0.75}

We see that the initial value of k must not be less than zero. This is also what was expected from the analysis of the verification.
\end{enumerate}

\item What could be gained by combined testing and proving?

