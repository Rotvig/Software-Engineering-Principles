\chapter{Problem 2}
When modelling with problem frames it is important not to commit to early solution. Problem frames are focused on the problem solving aspect of requirements modelling. 

\myFigure{figure/Context_Diagram.png}{Context diagram}{fig:context}{1}

The relation between the modules is explained below:

\begin{enumerate}[label=\alph*:]
	\item Heats the food
	\item Shows time left
	\item Handles the cooking time
	\item Alerts user when microwave is done
	\item Button that starts the microwave 
	\item Enter time, intensity and start microwave
	\item User input
	\item Open/close door
	\item Illuminates lamp when the microwave is started
	 
\end{enumerate}

To make the best model of the system, subproblem diagrams has been made. The subproblem diagrams is divided due to which requirements the diagram fulfills. 
The problem diagrams are listed below:

\myFigure{figure/Problem_diagram1.png}{Subproblem diagram 1}{fig:prob1}{0.8}

\myFigure{figure/Problem_diagram2.png}{Subproblem diagram 2}{fig:prob2}{0.8}

\myFigure{figure/Problem_diagram3.png}{Subproblem diagram 3}{fig:prob3}{0.7}

\myFigure{figure/Problem_diagram4.png}{Subproblem diagram 4}{fig:prob4}{0.7}


