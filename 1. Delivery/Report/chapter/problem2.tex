\chapter{Problem 2}

The Risk analysis in figure \ref{fig:risk} shows the most relevant risks for the project. For each risk the likelihood and consequences of that risk is rated from 1 to 10.

Short term illness for either an employee or child is very likely and almost expected. Therefore this is accounted for in the plan and the consequences are very low.

Long term illness is problematic because the team developing the product is very small. Long term illness of either an employee or a child of an employee is not very likely. The consequences are though very high. A solution for this is using freelance engineers but it will take time to introduce the project to a freelance engineer. Therefore, the consequences are very high. To try to reduce the risk all development must be well documented.

There is also a risk that an employee might quit. The project lasts in total 10 month and there is a small likelihood that an employee might quit. To try to ensure this will not happen its important to communicate with the staff and have regular meetings about job satisfaction. If an employee quits it will be handled in the same way as long term illness, namely by documenting well and using freelance engineers if necessary. 

A hardware supplier might also fail to deliver what have been agreed on for different reasons. To reduce this risk COTS hardware have mainly been chosen and the consequence is therefore not that high.

\myFigure{figure/riskAnalysis.png}{Risk analysis}{fig:risk}{1}