\chapter{Problem 6}

In the following the three models will be evaluated, to see what benefits and drawbacks they have. Also it will be evaluated if the models can be combined and if it can be beneficial.

\section{Problem Diagram}
The Context Diagram of Problem Frames is good for understanding the context in which a problem or system is set. This is illustrated by connecting all external blocks, that may affect your system, along with all internal blocks, to your system. The challenge is to keep it simple and not go into too much detail.

The Problem Frames specifically are related to the Context Diagram, and are therefore a good way of showing the requirements' relation to the rest of the system.


\section{Z-Notation}
The formal requirements by Z-notation is a good way of discovering lack of details in the requirements.  This means that it is very beneficial to use Z-notation, but there are some drawbacks as well. Z-notation is a complex model to understand, and can therefore require training. This could be a drawback.




\section{State Chart Diagram}
State charts are a way of making an overview of the different states in which a certain system can be. This is beneficial for getting an overview of the different possibilities of that system. It helps understanding and describing the behavior of the system in relation to the requirements.

\section{Combination}
As the three models offer different benefits and aspects of requirements, it will often be beneficial to use them together. In different projects different combinations might be more relevant.