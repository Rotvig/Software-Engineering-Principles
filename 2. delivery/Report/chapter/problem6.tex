\chapter{Problem 6}

In the following the three models will be evaluated, to see what benefits and drawbacks they have. Also it will be evaluated if the models can be combined and it will be beneficial.

\section{Problem Diagram}
The Context Diagram of Problem Frames is good for understanding the context in which a problem or system is set. This is illustrated by connecting all external blocks, that may affect your system, along with all internal blocks, to your system. The challenge is to keep it simple and not go into too much detail.

The Problem Frames specifically are relative to the Context Diagram, and are therefore a good way of showing the requirements relativity to the rest of the system.


\section{Z-Notation}
The formal requirements by Z-notation is a good way of discovering lack of details in the requirements. All details will be handled and every exception will be dealt with, by making new requirements. This means that it is very beneficial to use Z-notation, but there are some drawbacks as well. Reading a z-notation requires training from the reader. This cannot be done from all people. The amount of developers is now decreased to a small portion. This is a huge drawback of Z-notation. The time used to develop the requirements can also be a drawback.




\section{State Chart Diagram}
State charts are a way of making an overview of the different states in which a certain system can be. This is beneficial for getting an overview of the different possibilities of that system. However, state charts do not show a concrete example of a user scenario, which means it is not useful for illustrating specific scenarios.

State charts are also good for viewing your requirements relative to your system model.