\chapter{Problem 2}
The modelling with problem frames it is important not to commit to early solution. Before implementing, an understanding of the problem is important. Problem frames has a focus on the problem solving aspect of requirements modelling. 

\myFigure{figure/Context_Diagram.png}{Context diagram}{fig:context}{1}

The relation between the modules is explained below:

\begin{enumerate}[label=\alph*:]
	\item Heats the food
	\item Shows time left
	\item handles the timer
	\item Alerts user when microwave is done
	\item Button that starts the microwave 
	\item Enter food, enter time, starts microwave
	\item User input
	\item Open/close door
	\item Illuminates lamp when the microwave is started
	 
\end{enumerate}

To make the best model of the system, subproblem diagrams has been made. The subproblem diagrams is divided due to which requirements the diagram fulfils. 
The problem diagrams are listed below:

\myFigure{figure/Problem_diagram1.png}{Problem diagram 1}{fig:prob1}{0.8}

\myFigure{figure/Problem_diagram2.png}{Problem diagram 2}{fig:prob2}{0.8}

\myFigure{figure/Problem_diagram3.png}{Problem diagram 3}{fig:prob3}{0.7}

\myFigure{figure/Problem_diagram4.png}{Problem diagram 4}{fig:prob4}{0.7}


