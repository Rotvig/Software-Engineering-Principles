\chapter{Problem 1}
\label{chp:one}

%Carry out a harzard analysis for the microwave oven. What are the major hazards? How and why?

In this chapter, a hazard analysis is carried out. This will help the project be prepared for safety issues. 

\begin{center}
	\begin{tabular}{p{2.1cm} | p{2.9cm} | p{5cm} | p{4cm} }
		Guide Word & Deviation & Possible Causes & Possible Consequences \\\hline
		NONE & Containment loss & 
		\begin{enumerate}
			\item Magnetron tube starts even though door is not yet closed 
			\item Magnetron tube does not stop when door is opened 
		\end{enumerate}
		 & Magnetron tube's radiation will not be contained in the Microwave oven \\\hline
		
		NONE & High radiation & 
		\begin{enumerate}
			\item Intensity for radiation control goes over threshold 
		\end{enumerate}
		& Radiation leak becomes high enough to cause harm to user or others\\\hline
	\end{tabular}
\end{center}

\begin{description}
	\item[Containment loss] It is a safety concern if the magnetron tube starts when the door to the Microwave oven is not closed. This would mean the radiation that the magnetron tube emits would not be contained in the Microwave. Likewise it would be critical if the magnetron tube does not stop when the door is opened. For example if the microwave was heating food, and the user opens the door before the timer has run out. This could result in material damage and injury to users.
	
	\item[High radiation] Any microwave oven has a certain amount of radiation leakage. This leakage is proportional with the intensity for which the microwave heats the inserted object. If the intensity exceeds the threshold, which is the maximum intensity, the radiation could get high enough to be harmful outside of the microwave oven. 
\end{description}